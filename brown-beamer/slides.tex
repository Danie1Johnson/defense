\documentclass{beamer}
%\documentclass[handout,t]{beamer}
\newcommand{\btVFill}{\vskip0pt plus 1filll}

\batchmode
% \usepackage{pgfpages}
% \pgfpagesuselayout{4 on 1}[letterpaper,landscape,border shrink=5mm]

\usepackage{amsmath,amssymb,enumerate,epsfig,bbm,calc,color,ifthen,capt-of}

\usetheme{Berlin}
\usecolortheme{bear}

\title{Combinatorial and Geometric Structure in the Self-Assembly of Polyhedra}
\author{Daniel Johnson}
%\date{\today}
%\date{April 15, 2015}
\date{March 26, 2015}
\pgfdeclareimage[height=1cm]{brown-logo}{brown-logo.pdf}
\logo{\pgfuseimage{brown-logo}\hspace*{0.3cm}}

\AtBeginSection[]
{
  \begin{frame}<beamer>
    \frametitle{Outline}
    \tableofcontents[currentsection]
  \end{frame}
}
\beamerdefaultoverlayspecification{<+->}
% -----------------------------------------------------------------------------
\begin{document}
% -----------------------------------------------------------------------------

\frame{\titlepage}

\section[Outline]{}
\begin{frame}{Outline}
  \tableofcontents
\end{frame}
% -----------------------------------------------------------------------------
% -----------------------------------------------------------------------------
\section{Introduction}
\subsection{Polyhedra}
\begin{frame}{Platonic Solids}
\begin{itemize}
  \item Composed of a single type of regular polygon
  \item 5 Platonic Solids
\end{itemize}
\vspace{0.1 in}
\begin{columns}
    \begin{column}{0.32\textwidth}
      \centering
      %Cube

      \scalebox{0.035}{
        \begin{figure}
          \centering
          \includegraphics{cube.png}
        \end{figure}
      }

      Cube
     \end{column}
    \begin{column}{0.32\textwidth}
      \centering
      %Octahedron
      
      \scalebox{0.035}{
        \begin{figure}
          \centering
          \includegraphics{octahedron.png}
        \end{figure}
      }

      Octahedron
     \end{column}
    \begin{column}{0.32\textwidth}
      \centering
      
      \scalebox{0.035}{
        \begin{figure}
          \centering
          \includegraphics{dodecahedron.png}
        \end{figure}
      }

      Dodecahedron
     \end{column}
\end{columns}
\btVFill
\textit{\scriptsize Image credit: wikipedia users DTR, Stannered}
%<1->
\end{frame}
% -----------------------------------------------------------------------------
\begin{frame}{Archimedean Solids}
\begin{itemize}
  \item Composed of more than one type of regular polygon
  \item Identical vertices
  \item 13 Archimedean Solids
\end{itemize}
\vspace{0.1 in}
\begin{columns}
    \begin{column}{0.32\textwidth}
      \centering
      %Octahedron
      
      \scalebox{0.06}{
        \begin{figure}
          \centering
          \includegraphics{cuboctahedron.jpg}
        \end{figure}
      }

      Cuboctahedron
     \end{column}
    \begin{column}{0.32\textwidth}
      \centering
      %Cube

      \scalebox{0.06}{
        \begin{figure}
          \centering
          \includegraphics{truncated_octahedron.jpg}
        \end{figure}
      }

      Truncated Octahedron
     \end{column}
    \begin{column}{0.32\textwidth}
      \centering
      
      \scalebox{0.06}{
        \begin{figure}
          \centering
          \includegraphics{truncated_icosahedron.jpg}
        \end{figure}
      }

      Truncated Icosahedron
     \end{column}
\end{columns}
\btVFill
\textit{\scriptsize Image credit: wikipedia user Cyp}
\end{frame}
% -----------------------------------------------------------------------------
\begin{frame}{Catalan Solids}
\begin{itemize}
  \item Composed of a single type of non-regular polygon
  \item Dual to Archimedean solids
  \item 13 Catalan Solids
\end{itemize}
\vspace{0.1 in}
\begin{columns}
    \begin{column}{0.32\textwidth}
      \centering
      %Cube

      \scalebox{0.06}{
        \begin{figure}
          \centering
          \includegraphics{rhombicdodecahedron.jpg}
        \end{figure}
      }

      Rhombic Dodecahedron
     \end{column}
    \begin{column}{0.32\textwidth}
      \centering
      %Octahedron
      
      \scalebox{0.06}{
        \begin{figure}
          \centering
          \includegraphics{deltoidal_icositetrahedron.jpg}
        \end{figure}
      }

      Deltoidal Icositetrahedron
     \end{column}
    \begin{column}{0.32\textwidth}
      \centering
      
      \scalebox{0.06}{
        \begin{figure}
          \centering
          \includegraphics{pentagonal_hexecontahedron.jpg}
        \end{figure}
      }

      Pentagonal Hexecontahedron
     \end{column}
\end{columns}
\btVFill
\textit{\scriptsize Image credit: wikipedia users Cyberpunk and Maxim Razin}
\end{frame}
% -----------------------------------------------------------------------------
\subsection{Scientific Motivation}
\begin{frame}{Viral Capsids}
%\begin{columns}
%    \begin{column}{0.48\textwidth}
%
%    \end{column}
%    \begin{column}{0.48\textwidth}
%    \scalebox{0.19}{
%      \begin{figure}
%        \includegraphics{Adenovirus.jpg}
%        %\hspace*{15pt}\hbox{\scriptsize Credit:\thinspace{\small\itshape National Cancer Institute}}
%        %\caption{Credit: National Cancer Institute}
%       \end{figure}
%     }
%    \end{column}
%\end{columns}
  \centering
    \scalebox{0.25}{
      \begin{figure}
        \centering
        \includegraphics{Adenovirus.jpg}
       \end{figure}
     }
       %\caption{Credit: National Cancer Institute}
\btVFill
\textit{\scriptsize Image credit: National Cancer Institute} 
\end{frame}
% -----------------------------------------------------------------------------
\begin{frame}{Molecular Cages}
%\begin{columns}
%    \begin{column}{0.48\textwidth}
%
%    \end{column}
%    \begin{column}{0.48\textwidth}
%    \scalebox{0.1}{
%      \begin{figure}
%        \includegraphics{ward_cage.png}
%        %\hspace*{15pt}\hbox{\scriptsize Credit:\thinspace{\small\itshape National Cancer Institute}}
%        %\caption{Credit: National Cancer Institute}
%       \end{figure}
%      }
%    \end{column}
%\end{columns}
  \centering
    \scalebox{0.12}{
      \begin{figure}
        \centering
        \includegraphics{ward_cage.png}
       \end{figure}
      }
\btVFill
\textit{\scriptsize Image credit: Ward Research Group, NYU} 
\end{frame}
% -----------------------------------------------------------------------------
\begin{frame}{Self-Folding Polyhedra}
\end{frame}
% -----------------------------------------------------------------------------
\begin{frame}{Other Applications}
\begin{columns}
    \begin{column}{0.48\textwidth}
%        \includegraphics[width=<X>\textwidth]{}
\scalebox{0.19}{
        \includegraphics{jeremy_polyhedra.jpg}
}
    \end{column}
    \begin{column}{0.48\textwidth}
\scalebox{0.19}{
        \includegraphics{mr_jupiter.jpg}
}
    \end{column}
\end{columns}
\end{frame}
% -----------------------------------------------------------------------------
% -----------------------------------------------------------------------------
\section{The Building Game: Modeling}
\subsection{Definitions}
\begin{frame}{The Building Game}
%\begin{columns}
%    \begin{column}{0.48\textwidth}
%
%    \end{column}
%    \begin{column}{0.48\textwidth}
%    \scalebox{0.25}{
%      \begin{figure}
%        \includegraphics{bg.png}
%       \end{figure}
%      }
%    \end{column}
%\end{columns}
    \scalebox{0.5}{
      \begin{figure}
        \centering
        \includegraphics{bg.png}
       \end{figure}
      }
\begin{itemize}
  \item Begin with a single face.
  \item At each step, add a face adjacent to an already added face.
  \item End when all faces have been added.
\end{itemize}
\btVFill
\textit{\scriptsize Image credit: Govind Menon, Brown} 

\end{frame}
% -----------------------------------------------------------------------------
\begin{frame}{States and Intermediates}
\begin{definition}
  A Building Game \textbf{state} $x \subset F$ is a non-empty subset of the faces $F$ of a polyhedron such that the the subset is connected along edges. 
\end{definition} 
\begin{definition}
A Building Game \textbf{intermediate} $[x]$ is an equivalence class on states given by the equivalence relation: $x \sim \hat{x}$ if $x$ can be rotated to get $\hat{x}$.
\end{definition}

\end{frame}
% -----------------------------------------------------------------------------
\begin{frame}{Connections and Degeneracies}
\end{frame}
% -----------------------------------------------------------------------------
\begin{frame}{Configuration Space: Cube}
    \scalebox{0.4}{
      \begin{figure}
        \centering
        \includegraphics{cube_bg.png}
       \end{figure}
      }
\end{frame}
% -----------------------------------------------------------------------------
\begin{frame}{Configuration Space: Dodecahedron}
    \scalebox{0.4}{
      \begin{figure}
        \centering
        \includegraphics{dodecahedronSS.png}
       \end{figure}
      }
\end{frame}
% -----------------------------------------------------------------------------
\begin{frame}{Pathways}
\end{frame}
% -----------------------------------------------------------------------------
\subsection{Stochastic Modeling} 
\begin{frame}{Energetic Model}
\end{frame}
% -----------------------------------------------------------------------------
\begin{frame}{Markov Processes}
\begin{itemize}
\item $X_t$ a continuous time Markov process on the configurationa space graph.
\item Transition rates given by matrix $Q$
\end{itemize}
\begin{align}
  \label{eq:Qdef}
  Q_{jk} &=
  \begin{cases}
   S_{jk}e^{-\beta\left(E_{jk} - E_{j}\right)} & \text{if } [x^j] \leftrightarrow [x^k]  \\
   -z_j       & \text{if } j = k \\
   0 & \text{else}
  \end{cases} \\
z_j &= \sum_{\ell: \ell \neq j} S_{j\ell}e^{-\beta\left(E_{j\ell} - E_j\right)}
\end{align}
\end{frame}
% -----------------------------------------------------------------------------


\begin{frame}{Stationary Distribution}
\begin{theorem}
The Markov process $X_t$ with rate matrix $Q$ has stationary distribution $\pi$.
$$\pi_j &= \frac{1}{zr_j}e^{-\beta E_j}$$
\end{theorem}
\begin{lemma}
$$ r_jS_{kj} = r_{k}S_{jk}$$
\end{lemma}
\end{frame}
% -----------------------------------------------------------------------------
\begin{frame}{Stationary Distribution II}
\begin{proof}
Detailed Balance!
\begin{align}
\pi_jQ_{jk} &= \left(\frac{1}{zr_j}e^{-\beta E_j}\right)\left(S_{jk}e^{-\beta\left(E_{jk} - E_j\right)}\right) \\
&= \left(\frac{1}{z}e^{-\beta E_{jk}}\right)\left(\frac{S_{jk}}{r_j}\right) \\
&= \left(\frac{1}{z}e^{-\beta E_{kj}}\right)\left(\frac{S_{kj}}{r_k}\right) \\
&= \pi_kQ_{kj}
\end{align}
\end{proof}
\end{frame}
% -----------------------------------------------------------------------------
\begin{frame}{Formation Times}
$$\tau_{j} &\doteq \inf\left\{t \geq 0 : X_t = [F], X_0 = [x^j]\right\}$$
\end{frame}
% -----------------------------------------------------------------------------
% -----------------------------------------------------------------------------
\section{The Building Game: Enumeration}
\subsection{Configuration Space Enumeration}
\begin{frame}{Configuration Space Statistics}
\begin{figure}[ht]
\scalebox{0.5}{
%{\footnotesize
\centering
%\textbf{Building Game Enumerative Results for the Platonic Solids}
\begin{tabular}{ l | c | r | r | r}
Polyhedra Name & $|F|$ & Intermediates & Connections & Pathways \\
  \hline    
Tetrahedron                     & 4        & 4     	& 3             & 1\\
Cube                            & 6        & 8     	& 9    		& 3\\
Octahedron                      & 8        & 14    	& 21    	& 14\\
Dodecahedron                    & 12       & 73    	& 263   	& 17,696 \\
Icosahedron                     & 20       & 2,649 	& 17,241        & 57,396,146,640\\ \hline
Truncated Tetrahedron           & 8     & 28    	& 63            & 402\\
Cuboctahedron                   & 14  	& 340   	& 1,634         & 10,170,968\\
Truncated Cube                  & 14  	& 499   	& 2,729         & 101,443,338 \\
Truncated Octahedron            & 14  	& 555           & 3,069         & 68,106,377\\
Rhombicuboctahedron             & 26  	& 638,850       & 6,459,801     & 164,068,345,221,515,292,308\\
Truncated Cuboctahedron         & 26  	& 1,525,658     & 17,672,374    & 13,837,219,462,483,379,105,902\\ \hline  
Triakis Tetrahedron             & 12  	& 98            & 318           & 38,938\\
Rhombic Dodecahedron            & 12  	& 127           & 493           & 76,936\\
Triakis Octahedron              & 24  	& 12,748        & 81,296        & 169,402,670,046,670\\
Tetrakis Hexahedron             & 24  	& 50,767        & 394,377       & 4,253,948,297,210,346\\
Deltoidal Icositetrahedron      & 24  	& 209,675       & 1,989,548     & 418,663,242,727,526,726 \\
Pentagonal Icositetrahedron     & 24  	& 345,938       & 3,544,987     & 2,828,128,000,716,774,492\\
Rhombic Triacontahedron         & 30  	& 2,423,212     & 26,823,095    & 161,598,744,916,797,017,978,128\\
\end{tabular}
}
%\caption{Building game combinatorial configuration space enumerative results for the Platonic, Archimedean, and Catalan solids.}
\label{tab:bgEnum}
\end{figure}

\end{frame}
% -----------------------------------------------------------------------------

\begin{frame}{Computation}
\begin{itemize}
\item Given all intermediates with $k$ faces, compute the ones with $k+1$ faces.
\item Brute force approach necessary. 
\item Most of computation spent comparing two states for equivalence.
\item Smart hash function can reduce average comparison time by over an order of magnitude.
\end{itemize}
\end{frame}
% -----------------------------------------------------------------------------

\subsection{Shellings}
\begin{frame}{Shellings and Shellability}
\begin{definition}
A \textbf{shelling} is a linear ordering $f_1, f_2, \dots f_N$ of a polyhedra's faces such that 
$$f_j \cap \left(\bigcup_{i=1}^{j-1}f_i\right)$$ is connected for each $j$.
\end{definition}
\begin{theorem}
$$\#(\text{shellings})&= \sum_{p \in \{\text{shellable paths}\}}|[x^{p_1}]|\prod_j^{|F|-1}S_{p_jp_{(j+1)}}$$
\end{theorem}
\end{frame}

\begin{frame}{Shelling Enumeration}
\begin{figure}[ht]
\scalebox{0.6}{
%{\footnotesize
\centering
%\textbf{Building Game Enumerative Results for the Platonic Solids}
\begin{tabular}{ l | c | r}
Polyhedra Name & $|F|$ & Shellings \\
  \hline    
Tetrahedron                     & 4  	& 24 \\                         
Cube                            & 6  	& 480 \\                        
Octahedron                      & 8  	& 4,224\\                       
Dodecahedron                    & 12 	& 19,041,600 \\                 
Icosahedron                     & 20 	& 1,417,229,099,520 \\ \hline   
Truncated Tetrahedron           & 8     & 9,216 \\                      
Cuboctahedron                   & 14	& 113,055,744 \\                
Truncated Cube                  & 14	& 654,801,408 \\                
Truncated Octahedron            & 14	& 937,087,104 \\                
Rhombicuboctahedron             & 26	& 4,728,400,467,971,102,208 \\  
Truncated Cuboctahedron         & 26	& 688,499,026,944,479,645,952 \\ \hline
Triakis Tetrahedron             & 12    & 587,040\\                     
Rhombic Dodecahedron            & 12 	& 5,836,800\\                   
Triakis Octahedron              & 24	& 66,063,419,534,592 \\         
Tetrakis Hexahedron             & 24	& 1,389,323,257,015,296 \\      
Deltoidal Icositetrahedron      & 24	& 125,987,819,253,281,472\\     
Pentagonal Icositetrahedron     & 24	& 1,144,572,832,023,047,616 \\  
Rhombic Triacontahedron         & 30	& 15,574,782,555,813,226,074,240 \\
\end{tabular}
}
%\caption{Number of Shellings for the Platonic, Archimedean, and Catalan solids of up to 30 faces.}
\label{tab:Shellings}
\end{figure}
\end{frame}
% -----------------------------------------------------------------------------
\begin{frame}{Shelling Computation}
Dynamic programming relation! 
\begin{align}
a_{i,k} &\doteq \sum_{\substack{\text{shellable subpaths}:\\ p_1, \dots, p_k \\ x^{p_k} \in [x^i]}}|G.x^{p_1}|\prod_{j=1}^{k-1}S_{p_j p_{j+1}} \\  
&= \sum_{[x^\ell]:[x^\ell]\xrightarrow{shell}[x^i]} S_{\ell i} a_{\ell, k-1}
\end{align}

\end{frame}
% -----------------------------------------------------------------------------
% -----------------------------------------------------------------------------
\section{Conclusions}
\begin{frame}{Summary}
\end{frame}
% -----------------------------------------------------------------------------
\begin{frame}{Acknowledgements}
\end{frame}


\end{document}
% -----------------------------------------------------------------------------
% -----------------------------------------------------------------------------
% -----------------------------------------------------------------------------
% -----------------------------------------------------------------------------
% -----------------------------------------------------------------------------
\section{Introduction}
\subsection{Cracked Pots -- an uncharted field}
\begin{frame}{Why study Psychoceramics?}
  \begin{itemize}
    \item Plenty of subjects available for study
    \item Bright Job Prospects
    \item Ample Funding (Josiah S. Carberry Fund)
  \end{itemize}
\end{frame}
\begin{frame}{Past Work}
  Work done by JSC:
  \pause
  \begin{itemize}
    \item<2-> Archaic Greek Architectural Revetments in Connection with Ionian Philology
    \item<3-> Another Catullus to Another Lesbia
  \end{itemize}
  \pause[4]
  Work done by Others:
  \begin{itemize}
    \item<5-> Nothing
  \end{itemize}
  
\end{frame}
% -----------------------------------------------------------------------------
\section{Conclusions}
\subsection{Additional Aspects}
\begin{frame}{Questions and Answers}
  More Questions?

  \begin{itemize}
    \item Browse \url{http://tinyurl.com/43dwrt}.
    \item Contact my assistant Truman Grayson.
  \end{itemize}
  
\end{frame}
% -----------------------------------------------------------------------------


